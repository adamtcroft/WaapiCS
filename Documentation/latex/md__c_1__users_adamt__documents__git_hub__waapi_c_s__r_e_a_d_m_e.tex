{\bfseries \mbox{\hyperlink{namespace_waapi_c_s}{Waapi\+CS}}} is a C\# abstraction layer (or \char`\"{}wrapper\char`\"{}) for Audiokinetic Wwise\textquotesingle{}s W\+A\+A\+PI. It is intended to make W\+A\+A\+PI tremendously easier to use in C\# by removing the difficulties of W\+A\+MP and J\+S\+ON from the end user.

{\bfseries \mbox{\hyperlink{namespace_waapi_c_s}{Waapi\+CS}}} is made available to you under the M\+IT License. This means it is free, open source, customizable, and available for commercial use/distribution.

The only legal requirement is that you provide attribution (give me credit) in your game.

{\bfseries Please note that \mbox{\hyperlink{namespace_waapi_c_s}{Waapi\+CS}} is still under active development. Known issues (such as calls not working) can be found under the \char`\"{}\+Issues\char`\"{} tab.}

Code samples can be found within the solution under {\bfseries Sample Project}

\subsection*{Prerequisites}

To use \mbox{\hyperlink{namespace_waapi_c_s}{Waapi\+CS}}, you must have at least a beginner\textquotesingle{}s knowledge of C\#.

\mbox{\hyperlink{namespace_waapi_c_s}{Waapi\+CS}} was written using Visual Studio Community 2017. You can use it with other programming environments, but it (and the example code) work best with Visual Studio 2017.

\subsection*{\char`\"{}\+Installing\char`\"{} \mbox{\hyperlink{namespace_waapi_c_s}{Waapi\+CS}}}

To \char`\"{}install\char`\"{} and use \mbox{\hyperlink{namespace_waapi_c_s}{Waapi\+CS}}, download the repository using the big green \char`\"{}\+Clone or Download\char`\"{} button in the upper right hand corner.

Once the repository is on your machine, create a new project using Visual Studio 2017.

On the menu bar, then go to $\ast$\+\_\+\+Project $>$ Add Reference...\+\_\+$\ast$ then click the $\ast$\+\_\+\+Browse...\+\_\+$\ast$ button in the lower right corner of the window that pops up.

Navigate to the directory where you saved \mbox{\hyperlink{namespace_waapi_c_s}{Waapi\+CS}} and go to $\ast$\+\_\+\+Waapi\+CS$\ast$ and add {\itshape Waapi\+C\+S.\+dll}

Then you can begin using \mbox{\hyperlink{namespace_waapi_c_s}{Waapi\+CS}} in your code!

\subsection*{Using \mbox{\hyperlink{namespace_waapi_c_s}{Waapi\+CS}}}

The vast majority of \mbox{\hyperlink{namespace_waapi_c_s}{Waapi\+CS}}\textquotesingle{}s user-\/facing layer is made of \char`\"{}static calls\char`\"{} -\/ meaning you don\textquotesingle{}t create any objects like you normally would in C\#. A few commands stray from this convention, but not many.

I made a very large attempt to document the code (documentation comes with the respository download), and make the code itself mimic Wwise\textquotesingle{}s Authoring A\+PI itself.

For example, in W\+A\+A\+PI a call exists for {\ttfamily ak.\+wwise.\+core.\+get\+Info}, but you must write a lot of additional code to make it work. In \mbox{\hyperlink{namespace_waapi_c_s}{Waapi\+CS}}, this is simplified to {\ttfamily Dictionary$<$string,object$>$ results = \mbox{\hyperlink{classak_1_1wwise_1_1core_a0794d63ebaa3fc1a3c79d155978030ce}{ak.\+wwise.\+core.\+Get\+Info()}};}

Please refer to the included A\+PI documentation for how to use \mbox{\hyperlink{namespace_waapi_c_s}{Waapi\+CS}}. Where custom Wwise objects are referenced (such as object types or properties) you will need to refer to the original \href{https://www.audiokinetic.com/library/edge/?source=SDK&id=waapi__index.html}{\tt Wwise Authoring A\+PI reference}.

In some places you will need to use specific Wwise-\/related custom values. These can be found under the {\ttfamily Wwise\+Values} class, if available, by including {\ttfamily using \mbox{\hyperlink{namespace_waapi_c_s_1_1_custom_values}{Waapi\+C\+S.\+Custom\+Values}};} at the top of your code.

\subsection*{For Programmers}

\mbox{\hyperlink{namespace_waapi_c_s}{Waapi\+CS}} is built in two layers -\/ \mbox{\hyperlink{namespace_waapi_c_s}{Waapi\+CS}} and \mbox{\hyperlink{namespace_waapi_c_s_1_1_communication}{Waapi\+C\+S.\+Communication}}.

\mbox{\hyperlink{namespace_waapi_c_s_1_1_communication}{Waapi\+C\+S.\+Communication}} is the \char`\"{}lower\char`\"{} layer code, intended to communicate directly with Wwise via a packet system. \mbox{\hyperlink{namespace_waapi_c_s_1_1_communication}{Waapi\+C\+S.\+Communication}} houses a \char`\"{}packet\char`\"{} object, which is then passed to Wwise in J\+S\+ON form, returned back, and deserialized back into C\#.

\mbox{\hyperlink{namespace_waapi_c_s}{Waapi\+CS}} is the \char`\"{}higher\char`\"{} user-\/facing layer intended to make code writing easier for audio implementers and technical sound designers. In many cases, it will {\itshape not} be ideal for your studio to use this layer in its current state. Some of you will want to build custom objects to reflect the Wwwise hierarchy, for example.

I have intentionally separated \mbox{\hyperlink{namespace_waapi_c_s_1_1_communication}{Waapi\+C\+S.\+Communication}} and \mbox{\hyperlink{namespace_waapi_c_s}{Waapi\+CS}} layers so that you can easily customize or completely re-\/write the user facing layer without causing issues with the layer which directly communicates with Wwise.

Please feel free to make this your own, and contact me with any requests, suggestions, or updates.

\subsection*{If You Have Trouble}

If you need additional assistance, please get in touch with me via email -\/ \href{mailto:me@adamtcroft.com}{\tt me@adamtcroft.\+com} 